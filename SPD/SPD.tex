\documentclass[UTF8]{ctexart}   %指定文档的类型为 ctexart 使用 UTF-8 编码格式
\usepackage{ctex}               %调用ctex字库
\usepackage{graphicx}           %调用graphicx包库,用于插入图片

\setCJKmainfont{SimSun}         %设置主要字体为宋体(SimSun)

\newcommand{\stxs}{\CJKfamily{SimSun}\fontsize{12pt}{18pt}\selectfont}
                                %定义命令 \newcommand{cmd}{def} 
                                %\stxs 设定为使用中文宋体12磅1.5倍行间距
\title{
    \includegraphics[width=7.91cm,height=1.64cm]{logo.png}
                                %插入图片HBLGXY.png 设置图片宽7.91cm 高1.64cm
    \\
    \textbf{\CJKfamily{黑体}\fontsize{26pt}{39pt} 半导体物理与器件---名词解释}
                                %使用加粗黑体26磅1.5倍行距
}

\author{                        %作者信息
    背包客/落云\\                %这东西不重要之前用过的一用户名而已
    Apy6631@outlook.com         %显然是一份邮箱
}

\date{\today}                   %文本创作时间 \today 为机器当前时间

\begin{document}                %正文开启的类似头文件的东西

\maketitle                      %生成标题/封面 在此之前的至 \title{title} 之后的均为设计不会生成

\thispagestyle{empty}           %设置此页不显示页码


\clearpage                      %可以理解为分页符

\thispagestyle{empty}

\tableofcontents                %自动生成目录


\section{前言}                  %设置标题
                                %标题等级:由高到低(由大到小排序)
                                %\section{title}
                                %\subsection{title}
                                %\subsubsection{title}

\begin{center}
    写这个是为了练习一下LaTeX的使用,文本来自互联网及AI。
\end{center}
                                %\begin{center}
                                    %text
                                %\end{center}
                                %使文档text居中显示

\setcounter{page}{1}            % 将页码计数器重置为 1

\clearpage                      %同上分页符

\section{正文}

\subsection{半导体}

半导体:一种电阻介于导体和绝缘体之间的材料,具有光电、热电、半导电等特性,在现代电子技术中发挥着重要作用。

\subsection{空穴}

空穴:在半导体中,由于某些原因(如施主杂质)导致某些原子失去了一个或多个电子,从而形成了空位和少量带正电荷,这些空位被称为空穴。

\subsection{施主杂质}

施主杂质:指在半导体晶体中加入一些能够提供自由电子的杂质,使其成为n型半导体。常见的施主杂质包括磷、砷等元素。

\subsection{补偿半导体}

补偿半导体:一种杂质掺杂技术,通过在p型半导体中加入一定量的n型杂质(或在n型半导体中加入p型杂质),以调节半导体的电学性质,达到补偿作用。

\subsection{电离杂质散射}

电离杂质散射:电子在半导体中运动时与杂质原子相互作用,使电子的能量与方向发生改变的过程,会影响半导体的电导率等特性。

\subsection{冶金结}

冶金结:一种半导体器件结构,由金属和半导体构成,通过热扩散将金属与半导体形成均匀结合,具有较低的接触电阻和较高的电流密度。

\subsection{势垒电容}

势垒电容:由于p-n结两侧存在不同的耗尽区,导致p-n结上会形成一个能够存储电荷的电容,称为势垒电容。

\subsection{肖特基接触}

肖特基接触:由金属与半导体直接接触形成的接触,具有单向导电性和较低的接触电阻,常用于半导体器件的制造中。

\subsection{半导体功函数}

半导体功函数:描述半导体表面上电子从费米能级到真空能级所需的能量差,是表征半导体表面化学性质和电学性质的重要参数。

\subsection{杂质补偿半导体}

杂质补偿半导体:一种杂质掺杂技术,通过在p型半导体中同时加入n型和p型杂质,以达到精确控制半导体电学性质的目的。

\subsection{肖特基接触}

肖特基接触:由金属与半导体直接接触形成的接触,具有单向导电性和较低的接触电阻,常用于半导体器件的制造中。

\subsection{费米能级}

费米能级:描述在给定温度下,系统中所有粒子所填充的最高能级,是材料中电子分布状态的重要参考值。

\subsection{双极运输}

双极输运:指电荷在半导体中的移动过程,包括漂移和扩散两种方式。

\subsection{本征激发}

本征激发:半导体中自由电子受到外部电场或热激励等原因而跃迁至导带的现象,可形成载流子贡献到半导体的电导率中。

\subsection{电子有效质量}

电子有效质量:描述半导体中自由电子在晶格作用下的等效质量,是计算半导体物理特性时的重要参数。

\subsection{迁移率}

迁移率:描述材料中载流子在外加电场作用下的速度,是半导体材料电学性质的重要参考值。

\subsection{间接带隙半导体}

间接带隙半导体:指半导体材料中,导带和价带之间的能隙是通过晶格动量来实现的,如硅、锗等材料。

\subsection{欧姆定律}

欧姆接触:在金属与半导体或半导体与半导体接触区域内,存在着低阻态的接触,称为欧姆接触,通常用于制造半导体器件中。

\subsection{状态密度函数}

状态密度函数:描述材料中各个能级上可以容纳的电子数目,是计算半导体材料光学和电学特性时的重要参数。

\subsection{漂移电流}

漂移电流:指在电场作用下,载流子受到晶格作用而发生的类似于漂流的移动过程,是半导体器件中的重要电流成分。

\subsection{霍尔效应}

霍尔效应:描述在磁场作用下,半导体中载流子运动方向和磁场垂直时,会产生横向电势差的现象,称为霍尔效应,是半导体物理研究中的重要现象之一。

\subsection{扩散导电}

扩散电导:指在浓度差作用下,载流子从高浓度区域向低浓度区域扩散而形成的电流,是半导体器件中的重要电流成分。

\subsection{禁带}

禁带:指半导体材料中,价带和导带之间不允许任何能量状态的存在,这一区域称为禁带,决定了半导体材料导电特性的基本属性。

\subsection{受主掺杂}

受主掺杂:指在半导体晶体中加入一些能够吸收自由电子的杂质,使其成为p型半导体,常见的受主杂质包括硼、铝等元素。

\subsection{双极扩散}

双极扩散:指在半导体器件制造过程中,将掺有特定杂质的两个半导体片互相接触,并且通过热扩散将杂质原子在两个半导体中扩散,形成p-n结构,是制造半导体器件中重要的工艺步骤。

\clearpage                      %同上分页符

\section{结语}

\stxs{
     \hspace{2em}{此处首行缩进2字符}
     \hspace{2em}{连续使用是这种效果,然而}

     \hspace{2em}{空一空行继续使用会是这种效果。}

     此处首行缩进2字符连续使用是这种效果,然而空一空行继续使用会是这种效果。
                                %\stxs 为本文自定义的命令,详情请看上文|7行
                                %\hspace{size}{text}
                                %text文本首行缩进size个字符
                                %LaTeX中以换行判定分段 所以若连续使用时需要使用空行分段
}
\end{document}                  %结束document(正文)
